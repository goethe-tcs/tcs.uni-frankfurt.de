\documentclass[a4paper,11pt,oneside]{scrartcl}

% etwas mehr Platz für Feedback
\usepackage[right=6cm]{geometry}
\usepackage[onehalfspacing]{setspace}


% Standardpakete
\usepackage[utf8]{inputenc}
\usepackage[T1]{fontenc}
\usepackage{microtype}
\usepackage{amsmath}
\usepackage{booktabs}
\usepackage{tikz}
\usepackage{enumitem}
\usepackage{mathpazo}
\usepackage{graphicx}
\usepackage{hyperref}

% Sprache wählen
\usepackage[ngerman,shorthands=off]{babel}

% schöner Programmcode
\usepackage{listings}
\lstset{%
  language=C++,
  showstringspaces=false,
  mathescape=true,
  inputencoding=utf8,
  numbers=left,
  xleftmargin=\parindent,
  basicstyle=\footnotesize\ttfamily,
  keywordstyle=\bfseries\color{green!40!black},
  commentstyle=\normalfont\itshape\color{black!60},
  identifierstyle=\color{blue},
  stringstyle=\color{orange},
  tabsize=2%
}

\subtitle{ALGO1 --- SoSe 2021}

% Ausfüllen:
\title{Lösung zu Aufgabe x.y}

% Ausfüllen:
\author{%
  Alice Cooper (s0000001@stud.uni-frankfurt.de)%
  \and Bob Marley (s0000002@stud.uni-frankfurt.de)%
}

\begin{document}
\maketitle

\section{Aufgabenteil (a)}% falls es Aufgabenteile gibt.

In Aufgabenteil (a) ist zu zeigen, dass Algorithmus $A$ die Laufzeit $O(n^2)$ hat. Oder $\Omega(n\log n)$? Vielleicht auch $\Theta(n)$.

Sei $T(n)$ die Laufzeit von $A$ auf Eingaben der Größe $n$.
Dann gilt $T(n)\le T(n-1) + 10n$, denn $A$ macht genau zehn Rechenschritte und ruft sich dann selbst rekursiv wieder auf, mit einer Eingabe der Größe $n-1$. Wir beweisen nun mithilfe vollständiger Induktion, dass $T(n)\le 10n^2$ für alle positiven $n$ gilt:

\begin{itemize}
  \item Für den Induktionsanfang stellen wir fest, dass $T(1)=10=10 \cdot 1^2$ gilt.
  \item Für den Induktionsschluss, sei nun $n>1$. Wir können die Induktionsannahme vorrausetzen, dass $T(n-1)\le 10 (n-1)^2$ gilt, und müssen zeigen, dass $T(n)\le 10n^2$ gilt. Tatsächlich haben wir:
  \begin{align*}
    T(n) &\le T(n-1)+10n
    \le 10(n-1)^2 + 10n\\
    &= 10 (n^2-2n+1+n)
    = 10 (n^2-n+1) \le 10 n^2\,.
  \end{align*}
  Das war zu zeigen.
\end{itemize}

\section{Aufgabenteil (b)}

Bilder lassen sich mit \verb|\includegraphics{mein-bild.pdf}| einfügen, welches eine abphotographierte Zeichnung sein kann oder eine mit \href{https://inkscape.org/}{inkscape} produzierte Vektorgraphik. Ansonsten kann man auch das LaTeX-Paket tikz zum Zeichnen verwenden.

\section{Aufgabenteil (c)}

Programmcode oder Pseudocode kann so eingebunden werden:
\begin{lstlisting}[language=C++]
  for (int i = 1; i < n; i++) {
    for (int j = 1; j < i; j++) {
      std::cout << i; /* $\Theta(n^2)$ mal ausgefuehrt */
    }
  }
\end{lstlisting}
\end{document}
